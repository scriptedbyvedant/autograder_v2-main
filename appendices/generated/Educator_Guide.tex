\hypertarget{the-educators-guide-to-the-automated-grading-framework}{%
\section{The Educator's Guide to the Automated Grading
Framework}\label{the-educators-guide-to-the-automated-grading-framework}}

\begin{center}\rule{0.5\linewidth}{0.5pt}\end{center}

\hypertarget{purpose-scope}{%
\subsection{1. Purpose \& Scope}\label{purpose-scope}}

\textbf{Goal:} This guide serves as the primary onboarding and reference
document for educators using the Automated Grading Framework. Its
purpose is to empower you to efficiently and effectively grade student
assignments, understand the AI's reasoning, and improve its performance
over time.

\textbf{Scope:} This document covers the complete user workflow, from
uploading assignment materials and grading student submissions to
leveraging advanced features like model fine-tuning. It is not intended
to be a deep technical reference for developers.

\begin{center}\rule{0.5\linewidth}{0.5pt}\end{center}

\hypertarget{getting-started}{%
\subsection{2. Getting Started}\label{getting-started}}

\hypertarget{overview}{%
\subsubsection{2.1. Overview}\label{overview}}

The Automated Grading Framework is a tool designed to assist you with
the grading process. It uses a sophisticated AI Grading Engine to
analyze student submissions, provide a score based on your rubric, and
generate detailed, constructive feedback.

Crucially, \textbf{you are always in control}. The framework acts as
your expert assistant, and you have the final say on every grade. The
system is built on a \textbf{Human-in-the-Loop} philosophy, meaning your
corrections are not only saved but are used to make the AI smarter and
more aligned with your standards over time.

\hypertarget{prerequisites}{%
\subsubsection{2.2. Prerequisites}\label{prerequisites}}

Before you begin, please ensure you have the following:

\begin{itemize}
\tightlist
\item
  \textbf{Professor-level access credentials} to log into the
  application.
\item
  \textbf{Assignment materials in PDF format.} This includes:

  \begin{itemize}
  \tightlist
  \item
    A document containing the assignment questions, the ideal answers,
    and a detailed grading rubric.
  \item
    The students' submissions, also in PDF format.
  \end{itemize}
\end{itemize}

\begin{center}\rule{0.5\linewidth}{0.5pt}\end{center}

\hypertarget{step-by-step-walkthrough}{%
\subsection{3. Step-by-Step
Walkthrough}\label{step-by-step-walkthrough}}

This section will guide you through the three main phases of using the
application.

\hypertarget{phase-1-uploading-your-assignment}{%
\subsubsection{Phase 1: Uploading Your
Assignment}\label{phase-1-uploading-your-assignment}}

First, you need to provide the system with the context for the
assignment.

\begin{enumerate}
\def\labelenumi{\arabic{enumi}.}
\tightlist
\item
  \textbf{Navigate} to the \textbf{``Upload Data''} page from the main
  menu.
\item
  \textbf{Upload the Professor's Document:} Under the ``Upload Professor
  Data'' section, upload the PDF that contains your questions, ideal
  answers, and rubric.
\item
  \textbf{Upload Student Submissions:} Under the ``Upload Student Data''
  section, upload one or more student answer PDFs.
\item
  \textbf{Verification:} The system will confirm once the files are
  successfully parsed and stored.
\end{enumerate}

\begin{quote}
\mbox{}%
\hypertarget{best-practice-pdf-formatting}{%
\paragraph{\texorpdfstring{Note \textbf{Best Practice: PDF
Formatting}}{Note Best Practice: PDF Formatting}}\label{best-practice-pdf-formatting}}

For the best results, use PDFs where the text is machine-readable (i.e.,
not a scanned image). This allows the AI to parse the content with the
highest accuracy. If you have a very long assignment document, consider
splitting it into smaller files for easier processing.
\end{quote}

\hypertarget{phase-2-grading-and-review-the-human-in-the-loop}{%
\subsubsection{Phase 2: Grading and Review (The
Human-in-the-Loop)}\label{phase-2-grading-and-review-the-human-in-the-loop}}

This is where the magic happens. The AI will grade the submissions, and
you will review them.

\begin{enumerate}
\def\labelenumi{\arabic{enumi}.}
\tightlist
\item
  \textbf{Navigate} to the \textbf{``Grading Result''} page.
\item
  \textbf{Initiate Grading:} Select the course and assignment you wish
  to grade and click the \textbf{``Start Grading''} button.
\item
  \textbf{Review the Results:} After a few moments, the results will
  appear in an interactive table. For each student, you will see:

  \begin{itemize}
  \tightlist
  \item
    The question and their answer.
  \item
    The AI-generated score (\texttt{old\_score}) and feedback
    (\texttt{old\_feedback}).
  \end{itemize}
\item
  \textbf{Make Corrections:} If you disagree with the AI, simply
  \textbf{click into the table cell} and edit the score or feedback
  directly. The table works just like a spreadsheet.
\item
  \textbf{Save Corrections:} When you modify a grade, your correction is
  saved automatically as \texttt{new\_score} and \texttt{new\_feedback}.
\end{enumerate}

\begin{quote}
\mbox{}%
\hypertarget{how-your-corrections-help}{%
\paragraph{\texorpdfstring{  \textbf{How Your Corrections
Help}}{  How Your Corrections Help}}\label{how-your-corrections-help}}

Every correction you make is used in two powerful ways: 1. \textbf{For
Consistency (RAG):} Your correction is immediately stored in the
system's ``memory'' (a Vector Store). When the AI grades the \emph{next}
student, it looks at this memory to see how you graded similar answers,
helping it stay consistent. 2. \textbf{For Long-Term Improvement:} Your
corrections become the training data for making the AI model itself
better (see Phase 3).
\end{quote}

\hypertarget{phase-3-advanced-improving-the-ai-with-the-finetuning-assistant}{%
\subsubsection{Phase 3 (Advanced): Improving the AI with the Finetuning
Assistant}\label{phase-3-advanced-improving-the-ai-with-the-finetuning-assistant}}

After you have graded several assignments and made corrections, you can
use that data to create a new, smarter version of the AI model that is
customized to your specific course and standards.

\begin{enumerate}
\def\labelenumi{\arabic{enumi}.}
\tightlist
\item
  \textbf{Navigate} to the \textbf{``Model Finetuning Assistant''} page.
\item
  \textbf{Step 1: Generate Data:} Click the \textbf{``Generate Training
  Data''} button. The system will package all the corrections you've
  made into a single \texttt{training\_dataset.jsonl} file. A download
  button will appear.
\item
  \textbf{Step 2: Train in Colab:}

  \begin{itemize}
  \tightlist
  \item
    Follow the link to open Google Colab and set the runtime to
    \texttt{T4\ GPU} as instructed.
  \item
    Copy the provided Python script into a Colab cell.
  \item
    Upload your \texttt{training\_dataset.jsonl} file to the Colab
    environment.
  \item
    Run the cell. The training will take 15-20 minutes.
  \end{itemize}
\item
  \textbf{Step 3: Deploy Your Model:}

  \begin{itemize}
  \tightlist
  \item
    Once training is complete, a \texttt{trained\_adapters.npz} file
    will appear in Colab. Download it.
  \item
    Move this file into the \texttt{training/} folder of this
    application.
  \item
    \textbf{Restart the application.}
  \end{itemize}
\end{enumerate}

That's it! The application will automatically detect and use your new,
fine-tuned model for all future grading tasks.

\begin{center}\rule{0.5\linewidth}{0.5pt}\end{center}

\hypertarget{troubleshooting-faq}{%
\subsection{4. Troubleshooting \& FAQ}\label{troubleshooting-faq}}

\begin{itemize}
\tightlist
\item
  \textbf{Q: The PDF upload failed or the text looks jumbled. Why?}

  \begin{itemize}
  \tightlist
  \item
    \textbf{A:} This usually happens if the PDF is a scanned image of a
    document. Please ensure your PDFs are created from a text source
    (e.g., ``Save as PDF'' from a word processor). See the Best Practice
    tip in Phase 1.
  \end{itemize}
\item
  \textbf{Q: The AI's grade seems completely wrong. What should I do?}

  \begin{itemize}
  \tightlist
  \item
    \textbf{A:} Simply correct it in the results table. The system is
    designed for this! Your correction provides a valuable data point
    that helps the AI learn.
  \end{itemize}
\item
  \textbf{Q: How can I trust the AI is being fair and consistent?}

  \begin{itemize}
  \tightlist
  \item
    \textbf{A:} The system uses two key features for this: the
    \textbf{Multi-Agent System}, where multiple AI agents debate to
    reach a consensus, and \textbf{Retrieval Augmented Generation
    (RAG)}, which constantly refers to your past corrections to maintain
    consistency. The final authority, however, is always you.
  \end{itemize}
\item
  \textbf{Q: I deployed a new fine-tuned model but I want to revert to
  the original. How?}

  \begin{itemize}
  \tightlist
  \item
    \textbf{A:} Simply delete the \texttt{trained\_adapters.npz} file
    from the \texttt{training/} directory and restart the application.
    The system will revert to the base model.
  \end{itemize}
\end{itemize}

\begin{center}\rule{0.5\linewidth}{0.5pt}\end{center}

\hypertarget{additional-resources}{%
\subsection{5. Additional Resources}\label{additional-resources}}

For users interested in the underlying technical details of the
framework, the following documents are available in the project
repository:

\begin{itemize}
\tightlist
\item
  \texttt{ARCHITECTURE.md}: A high-level overview of the system
  components.
\item
  \texttt{DESIGN.md}: A detailed technical design of the software.
\item
  \texttt{DATA\_FLOW.md}: A set of diagrams illustrating how data moves
  through the application.
\end{itemize}

\begin{center}\rule{0.5\linewidth}{0.5pt}\end{center}

\hypertarget{next-steps-for-this-document}{%
\subsection{Next Steps for This
Document}\label{next-steps-for-this-document}}

As a living document, this guide will evolve with the product. Based on
anticipated user feedback, future versions should include:

\begin{itemize}
\tightlist
\item
  \textbf{A section on interpreting the Analytics Dashboard.}
\item
  \textbf{Specific guidance for grading Code Assignments}, including how
  to write effective \texttt{unittest} cases.
\item
  \textbf{A more detailed ``Tips and Tricks'' section} for writing
  effective rubrics that the AI can easily understand.
\end{itemize}
